\documentclass[11pt]{article}

\usepackage[margin=1in]{geometry}
\usepackage{amsmath,amsthm,amssymb}
\title{FEM and Inverse notes}
\author{Tobias Harvey}
\date{July 2024}
\begin{document}
\maketitle

\section{1D Variable Coefficent Diffusion}

$$s(x, t)\frac{\partial u}{\partial t} = \left(t(x,t)\frac{\partial u}{\partial x}\right)_x + q(x,t), \qquad 0 \leq x \leq L, t \leq T$$

$$u(0,t) = \alpha(t)$$

$$u(L,t) = \beta(t)$$

Assuming an approximate solution $u^h$, residual $u^h - u$ can be calculated by moving everything to the LHS and plugging in the approximation:

$$u^h - u = s(x, t)\frac{\partial u^h}{\partial t} - \left(t(x,t)\frac{\partial u^h}{\partial x}\right)_x - q(x,t)$$

enforce that error is not in test function space:

$$\int_0^L s(x, t)\frac{\partial u^h}{\partial t}V(x) dx -  \int_0^L \left(t(x,t)\frac{\partial u^h}{\partial x}\right)_x V(x) dx - \int_0^L q(x,t) V(x) dx = 0$$

Integration by parts:

$$\int_0^L s(x, t)\frac{\partial u^h}{\partial t}V(x) dx +  \int_0^L t(x,t)\frac{\partial u^h}{\partial x}\right \frac{\partial V(x)}{\partial x} dx -n t(x,t)\frac{\partial u^h}{\partial x} V(x)\bigg\rvert_0^L- \int_0^L q(x,t) V(x) dx = 0$$

make space of $V(x)$ finite and expand $u^h = \sum_{j=0}^N \hat{u}_j \psi_j$ in the $\psi$ basis and make $N$ equations for $N$ basis functions:

$$\int_0^L s(x, t)\sum_{j=0}^N \frac{\partial \hat{u}_j}{\partial t} \psi_j\psi_i dx -  \int_0^L t(x,t)\sum_{j=0}^N \hat{u}_j\frac{\partial \psi_j}{\partial x}\right \frac{\partial \psi_i}{\partial x} dx +   =  t(L,t)\beta(t)\psi_{pN}(L) - t(0,t)\alpha(t)\psi_{0}(0) + \int_0^L q(x,t) \psi_i dx$$

this is for a single local element $k$:

$$M^k\hat{u}_t - S^k\hat{u} = F^k$$

$$M^k_{ij} = \int_k s(x, t)\psi_j\psi_i \quad S^k_{ij} = \int_k t(x,t)\frac{\partial \psi_j}{\partial x}\right \frac{\partial \psi_i}{\partial x} dx$$


\section{2D Variable Coefficent Diffusion}

$$s(x, y, t)\frac{\partial u}{\partial t} = \left(t(x, y, t)\frac{\partial u}{\partial x}\right)_x + \left(t(x, y, t)\frac{\partial u}{\partial y}\right)_y + q(x, y, t), \qquad 0 \leq x \leq L_x, 0 \leq y \leq L_y, t \leq F$$

% TODO: FIX boundary conditions

$$u(0, y, t) = \alpha(t)$$

$$u(L,t) = \beta(t)$$

Assuming an approximate solution $u^h$, residual $u^h - u$ can be calculated by moving everything to the LHS and plugging in the approximation:

$$u^h - u = s(x, y, t)\frac{\partial u^h}{\partial t} - \left(T(x, y, t)\frac{\partial u^h}{\partial x}\right)_x -\left(t(x, y, t)\frac{\partial u}{\partial y}\right)_y - q(x, y, t)$$

enforce that error is not in test function space:

$$\int_\Omega s(x, y, t)\frac{\partial u^h}{\partial t}V(x) dx -  \int_\Omega \left(T(x, y, t)\frac{\partial u^h}{\partial x}\right)_x V(x) dx - \int_\Omega q(x,t) V(x) dx = 0$$

Integration by parts (greens identity):

\begin{equation*}
  \begin{split}
    & \int_\Omega s(x, y, t)\frac{\partial u^h}{\partial t}V(x) dx +  \int_\Omega T(x, y, t)\frac{\partial u^h}{\partial x}\right \frac{\partial V(x)}{\partial x} dx -
    \int_\Gamma (\boldsymbol{n_x} \cdot \boldsymbol{n_k}) T(x, y, t)\frac{\partial u^h}{\partial x} V(x)+ \\
    & \int_\Omega T(x, y, t)\frac{\partial u^h}{\partial y}\right \frac{\partial V(x)}{\partial y} dx - \int_\Gamma (\boldsymbol{n_y} \cdot \boldsymbol{n_k}) T(x, y, t)\frac{\partial u^h}{\partial y} V(x)- \\
    & \int_\Omega q(x,t) V(x) dx = 0
  \end{split}
\end{equation}

make space of $V(x)$ finite and expand $u^h = \sum_{j=0}^N \hat{u}_j \psi_j$ in the $\psi$ basis and make $N$ equations for $N$ basis functions:

\begin{equation*}
  \begin{split}
    &\int_\Omega s(x, y, t)\sum_{j=0}^N \frac{\partial \hat{u}_j}{\partial t} \psi_j\psi_i dx +  \int_\Omega T(x, y, t)\sum_{j=0}^N \hat{u}_j\frac{\partial \psi_j}{\partial x}\right \frac{\partial \psi_i}{\partial x} dx + \int_\Omega T(x, y, t)\sum_{j=0}^N \hat{u}_j\frac{\partial \psi_j}{\partial y}\right \frac{\partial \psi_i}{\partial y} dx =\\
    &\text{boundary terms } + \int_\Omega q(x,t) \psi_i dx\\
\end{equation*}
  \end{split}

this is for a single local element $k$:

$$M^k\hat{u}_t + S^k_x\hat{u} + S^k_y\hat{u} = F^k$$

$$M^k_{ij} = \int_k s(x, y, t)\psi_j\psi_i \quad S^k_x_{ij} = \int_k T(x, y, t)\frac{\partial \psi_j}{\partial x}\right \frac{\partial \psi_i}{\partial x} dx \quad S^k_y_{ij} = \int_k T(x, y, t)\frac{\partial \psi_j}{\partial y}\right \frac{\partial \psi_i}{\partial y} dx$$

\section{heat 1st order system?}

Alternatively split heat equation into first order system so that:

$$s(x, y, t)\frac{\partial u}{\partial t} = \left(t(x, y, t)\frac{\partial u}{\partial x}\right)_x + \left(t(x, y, t)\frac{\partial u}{\partial y}\right)_y + q(x, y, t), \qquad 0 \leq x \leq L_x, 0 \leq y \leq L_y, t \leq F$$


turns into:
\begin{equation*}
  \begin{split}
    s(x, y, t)\frac{\partial u}{\partial t} = \left(\sqrt{t}q)_x + \left(t(x, y, t)\frac{\partial u}{\partial y}\right)_y + q(x, y, t)
  \end{equation}
\end{split}


\section{basis and matrix construction}

Reference element: (-1, -1), (1, -1), (-1, 1) in the $r, s$ plane. Transformation from the $r,s$ plane to the $x,y$ plane is:

$$T(r,s) = \frac{r+s}{2}v^1 + \frac{r+1}{2}v^2 + \frac{s+1}{2}v^3$$
where $v^1, v^2, v^3$ are the vertices of the element in the $x,y$ plane.

jacobian of transformation is:
$$J = \frac{(v^2_x - v^1_x)(v^3_y - v^1_y)}{4} - \frac{(v^2_y - v^1_y)(v^3_x - v^1_x)}{4}$


Assuming linear basis functions (lame) they are on the reference element:

\begin{equation*}
  \begin{split}
    & \psi_1 = \frac{1}{2}s + \frac{1}{2} \\
    & \psi_2 = \frac{1}{2}r + \frac{1}{2} \\
    & \psi_3 = -\frac{1}{2}r - \frac{1}{2}s \\
  \end{split}
\end{equation}

derivaties are:

\begin{equation*}
  \begin{split}
    \frac{\partial\psi_1}{\partial r} = 0 & \quad \frac{\partial\psi_1}{\partial s} = \frac{1}{2} \\
    \frac{\partial\psi_2}{\partial r} = \frac{1}{2} & \quad \frac{\partial\psi_2}{\partial s} = 0 \\
        \frac{\partial\psi_3}{\partial r} = -\frac{1}{2} & \quad \frac{\partial\psi_3}{\partial s} = -\frac{1}{2} \\
  \end{split}
\end{equation}


Matrices need to be computed with numerical integration because of variable coefficents:

gaussian quad weights:

\begin{tabular}{|c|c|c|}
\hline
  r_i & s_i & w_i\\
\hline
1/3 & 1/3 & −9/32\\
3/5 & 1/5 & 25/96\\
1/5 & 3/5 & 25/96\\
1/5 & 1/5 & 25/96\\
\hline
\end{tabular}


\newpage

\subsection {linear local mass computation}

assume $\psi_i$ is linear
how to compute:

\begin{equation*}
  \begin{split}
    \begin{bmatrix}
      \int_k\psi_1 s \psi_1 d\Omega_k & \int_k\psi_1 s \psi_2 d\Omega_k & \int_k\psi_1 s \psi_3 d\Omega_k \\
      \int_k\psi_2 s \psi_1 d\Omega_k & \int_k\psi_2 s \psi_2 d\Omega_k & \int_k\psi_2 s \psi_3 d\Omega_k \\
            \int_k\psi_3 s \psi_1 d\Omega_k & \int_k\psi_3 s \psi_2 d\Omega_k & \int_k\psi_3 s \psi_3 d\Omega_k \\
    \end{bmatrix} = \\
    \begin{bmatrix}
      \sum_q\psi_1(x_q, y_q) s(x_q, y_q) \psi_1(x_q, y_q) w_q & \sum_q\psi_1(x_q, y_q) s(x_q, y_q)\psi_2(x_q, y_q) w_q & \sum_q\psi_1(x_q, y_q) s(x_q, y_q)\psi_3(x_q,y_q) w_q \\
      \sum_q\psi_2(x_q, y_q) s(x_q, y_q)\psi_1(x_q, y_q) w_q & \sum_q\psi_2(x_q, y_q) s(x_q, y_q)\psi_2(x_q, y_q) w_q & \sum_q\psi_2(x_q, y_q) s(x_q, y_q)\psi_3(x_q,y_q) w_q \\
      \sum_q\psi_3(x_q,y_q) s(x_q, y_q)\psi_1(x_q, y_q) w_q & \sum_q\psi_3(x_q,y_q) s(x_q, y_q)\psi_2(x_q, y_q) w_q & \sum_q\psi_3(x_q,y_q) s(x_q, y_q)\psi_3(x_q,y_q) w_q \\
    \end{bmatrix} = \\
  \end{split}
\end{equation}

\subsection{timestepping}

everything boils down to:
\begin{equation*}
  \begin{split}
    M\hat{u}_t = S\hat{u} + F
  \end{split}
\end{equation}

\noindent doing crank:

\begin{equation*}
  \begin{split}
    & M\frac{\hat{u}_{t+1} - \hat{u}_{t}}{\Delta t} = \frac{1}{2}(S\hat{u}_{t+1} + S\hat{u}_{t}) + \frac{1}{2}(F_{t+1} + F_{t})\\
   &  M\hat{u}_{t+1} - \frac{{\Delta t}}{2}S\hat{u}_{t+1} = M\hat{u}_{t} + \frac{{\Delta t}}{2}S\hat{u}_{t} + \frac{{\Delta t}}{2}F_{t+1} + \frac{{\Delta t}}{2}F_{t}\\
  \end{split}
\end{equation}




\end{document}
